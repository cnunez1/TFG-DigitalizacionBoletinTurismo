\apendice{Anexo de sostenibilización curricular}

\section{Introducción}

En este apéndice se explica la relación que existe entre el proyecto y los Objetivos de Desarrollo Sostenible (ODS) y las competencias de sostenibilidad adquiridas durante su desarrollo.
Para ello, se da una breve explicación de los ODS correspondientes y se relacionan con las tareas o funcionalidades realizadas en el proyecto.

Estos ODS fueron aprobados en 2015 por todos los miembros de la Organización de las Naciones Unidas (ONU) y tienen como objetivo 
acabar con la pobreza, mejorar la salud y la educación, reducir las desigualdades, garantizar la paz y prosperidad de la gente y del planeta e impulsar el crecimiento económico.\cite{ods}

\section{Objetivos de Desarrollo Sostenible}

El proyecto se centra en la sostenibilidad de los recursos turísticos de la provincia de Burgos, lo que implica una gestión responsable y sostenible de estos recursos.

\subsection{ODS 5: Igualdad de género}

Se proporcionan gráficas que permiten diferenciar las valoraciones y número de reseñas de los diferentes recursos dependiendo del género de los usuarios.
Esto permite identificar si hay alguna diferencia significativa en la valoración de los recursos turísticos entre hombres y mujeres, lo que puede ayudar a mejorar la oferta turística y adaptarla a las necesidades de ambos géneros.
También puede ayudar a identificar los recursos más valorados por un género u otro y que puede llevar a fomentar la visibilidad de los colectivos poco representados.

\subsection{ODS 9: Industria, innovación e infraestructura}

El desarrollo de este proyecto ha implicado el uso de tecnologías innovadoras como una red neuronal en relación al aprendizaje automático.
Además, la explotación de los datos obtenidos de forma pública permite mejorar la infraestructura turística de la provincia de Burgos.

\subsection{ODS 11: Ciudades y comunidades sostenibles}

Este proyecto contribuye a la sostenibilidad de las ciudades y comunidades al proporcionar información sobre los recursos de la provincia de Burgos.
A través del cuadro de mando, se puede identificar y analizar las zonas con un determinado tipo de recurso lo que permite detectar zonas mal valoradas o con pocos recursos que podrían ser mejoradas.

\section{Competencias de sostenibilidad adquiridas}

A través del desarrollo de este proyecto, he adquirido competencias de sostenibilidad definidas por la Conferencia de Rectores y Rectoras de las Universidades Españolas (CRUE)\cite{crue}.

\subsection{SOS1 - Competencia en la contextualización crítica del conocimiento estableciendo interrelaciones
con la problemática social, económica y ambiental, local y/o global}

Durante el desarrollo del proyecto se ha prestado atención al contexto social y territorial de la provincia de Burgos, 
entendiendo cómo la distribución de los puntos de interés (POIs) y la interacción de los ciudadanos con estos espacios a través de reseñas puede reflejar desigualdades económicas, concentración de servicios y posibles carencias estructurales. 

\subsection{SOS2 - Competencia en la utilización sostenible de recursos y en la prevención de impactos negativos sobre el medio natural y social}

El proyecto ha utilizado datos públicos de Google Maps, lo que implica un uso sostenible de los recursos digitales disponibles.
Además, en todo momento se ha tratado de optimizar y minimizar el impacto ambiental del proyecto, evitando el uso de recursos innecesarios y fomentando la reutilización de datos existentes.

\subsection{SOS4 - Competencia en la aplicación de principios éticos relacionados con los valores de la
sostenibilidad en los comportamientos personales y profesionales}

En relación a la ética, se ha procurado garantizar la transparencia en el uso de los datos y en la presentación de los resultados, evitando cualquier tipo de manipulación o tergiversación de la información.

\section{Conclusiones}

En conclusión, he aprendido a relacionar el proyecto con los ODS y a reflexionar sobre la sostenibilidad de los recursos turísticos de la provincia de Burgos.
He adquirido competencias de sostenibilidad que me permiten entender la importancia de la sostenibilidad en el desarrollo de proyectos y en la gestión de recursos turísticos.