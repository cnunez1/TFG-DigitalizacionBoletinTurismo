\apendice{Plan de Proyecto Software}

\section{Introducción}

En este apéndice se pretende detallar el plan de proyecto software seguido para el desarrollo del mismo. 
Para ello, se mostrará la planificación temporal, el estudio de viabilidad que comprende aspectos relevantes de su viabilidad económica y legal.

\section{Planificación temporal}

El proyecto se ha desarrollado siguiendo el modelo de desarrollo ágil SCRUM. Para ello, se han definido una serie de sprints de 2 semanas utilizando GitHub Projects.
En cada sprint se ha tratado de implementar una serie de funcionalidades definidas previamente como issues.

\subsection{Sprint 1 (27/02/2025 - 12/03/2025)}

Este primer sprint se centró en la creación del proyecto y en su inicio. Dado que este proyecto se basa en un TFG anterior, algunas tareas están relacionadas con ellas.

\begin{itemize}
    \item Leer y analizar el TFG previo relacionado con el tema.
    \item Definir el enfoque y tecnologías a utilizar en el proyecto.
    \item Probar diferentes APIs (de extracción de datos).
    \item Probar funcionalidades de Power BI.
    \item Probar extracciones de reviews a partir de GeoJSON de OpenStreetMaps.
    \item Probar funcionalidades de PowerPages.
    \item Probar la viabilidad de extraer datos con Apify.
    \item Probar Google Maps Scraper.
\end{itemize}

\subsection{Sprint 2 (13/03/2025 - 26/03/2025)}

En el segundo sprint, se trató de iniciar el proceso de documentación del proyecto y de iniciar el sistema de extracción de datos.

\begin{itemize}
    \item Obtener place ids sin usar los cuadrantes de las coordenadas.
    \item Iniciar el proceso de documentación
    \item Obtener place ids a partir de textSearch.
    \item Documentar los objetivos del proyecto.
    \item Actualizar documentación de los objetivos del proyecto.
    \item Automatizar la extracción de los datos del INE a través de su API. (Cerrada como no planeada)
\end{itemize}

\subsection{Sprint 3 (27/03/2025 - 09/04/2025)}

Para el tercer sprint, el objetivo principal era diseñar el modelo de datos, el cual fue actualizado en los últimos sprints.

\begin{itemize}
    \item Diseñar un esquema para el modelo de datos.
    \item Añadir manualmente los datos (poblacion y coordenadas) de los municipios que no provee el INE. (Cerrada como no planeada)
\end{itemize}

\subsection{Sprint 4 (10/04/2025 - 23/04/2025)}

En el cuarto sprint, se inició a desarrollar el sistema de aprendizaje automático y el primer modelo de red neuronal.

\begin{itemize}
    \item Crear una red neuronal a partir de reviews.
\end{itemize}

\subsection{Sprint 5 (24/04/2025 - 07/05/2025)}

En el quinto sprint se trató de crear el modelo final de extracción de datos.

\begin{itemize}
    \item Probar la extracción de reviews mediante Overpass QL, Google Places API y Apify.
    \item Crear una base de datos para almacenar las reviews.
\end{itemize}

\subsection{Sprint 6 (08/05/2025 - 21/05/2025)}

Durante el sexto sprint, tras no haber obtenido buenos resultados con el modelo de red neuronal, se optó por construir un dataset propio para el modelo de aprendizaje automático.

\begin{itemize}
    \item Crear un dataset con reviews y categorías de los POIs.
\end{itemize}

\subsection{Sprint 7 (22/05/2025 - 04/06/2025)}

El séptimo sprint se basó en la creación de un diccionario para mapear subcategorías a categorías para mejorar el modelo de aprendizaje automático.
Finalmente esto fue descartado tras rediseñar el conjunto de datos.

\begin{itemize}
    \item Hacer un mapeo de categorías a subcategorías (Cerrada como no planeada).
\end{itemize}

\subsection{Sprint 8 (05/06/2025 - 18/06/2025)}

El octavo sprint se centró en redactar la documentación de la memoria del proyecto. El objetivo era crear una versión inicial pero completa de la memoria.

\begin{itemize}
    \item Añadir conceptos teóricos a la documentación.
    \item Añadir técnicas y herramientas a la documentación.
    \item Añadir aspectos relevantes a la documentación.
    \item Añadir los trabajos relacionados a la documentación.
\end{itemize}

\subsection{Sprint 9 (19/06/2025 - 02/07/2025)}

El penúltimo sprint se centró en completar la memoria y crear el cuadro de mando.

\begin{itemize}
    \item Añadir la introducción a la documentación.
    \item Añadir las conclusiones y líneas de trabajo futuras a la documentación.
    \item Crear un dashboard de Power BI para la visualización de resultados.
    \item Crear una Azure Function para la predicción y guardados automáticos sobre los recursos.
\end{itemize}

\subsection{Sprint 10 (03/07/2025 - 07/07/2025)}

En el último sprint se ha tratado de completar toda la documentación y el cuadro de mando.
Además se ha ordenado el repositorio y se han actualizado los ficheros antiguos.

\begin{itemize}
    \item Implementar un sistema de análisis y detección de defectos de código.
    \item Crear un sistema de recomendación a partir de la localización y reseñas de los recursos.
    \item Añadir el plan de proyecto a los anexos.
    \item Añadir los requisitos a los anexos.
    \item Añadir la especificación de diseño a los anexos.
    \item Añadir el manual de programador a los anexos.
    \item Añadir el manual de usuario a los anexos.
    \item Añadir el anexo de sostenibilización curricular.
\end{itemize}

\imagen{commits}{Gráfica de commits}{1}

\section{Estudio de viabilidad}

\subsection{Viabilidad económica}

Desde el punto de vista económico, hay varios aspectos a tener en cuenta:

\subsubsection{Costes de hardware}

Este proyecto ha sido desarrollado en un ordenador de sobremesa con un coste aproximado de 800 euros comprado hace 9 años por lo que se puede considerar amortizado.

\subsubsection{Extracción de datos}

Para la extracción de datos se ha optado por utilizar la plataforma Apify que permite extraer datos de forma masiva de páginas web y APIs.
Se muestra una tabla con los precios de las cuentas de Apify\cite{apify:pricing} en dólares américanos:
\begin{table}[H]
    \centering
    \begin{tabular}{|c|c|c|c|c|}
        \hline
        \textbf{Tipo de cuenta} & \textbf{Número de reviews} & \textbf{Precio/mes al 50\%} \\
        \hline
        Free & 14285 & 0\$ \\
        Starter & 111428 & 19,5\$ \\
        Scale & 568571 & 99,5\$ \\
        Business & 2854285 & 499,5\$ \\
        Enterprise & Ilimitado & Hablar con Apify \\
        \hline
    \end{tabular}
    \caption{Costos por solicitud en Apify}
\end{table}
Cabe destacar que ofrecen un descuento del 50\% para cuentas educativas.

Se estiman alrededor de un millón y medio de reseñas en toda la provincia de Burgos, por lo que la mejor opción puede parecer la cuenta Business.
Sin embargo, hay muchos recursos con pocas reseñas o sin reseñas que podrían ser descartados por lo que la mejor opción serían 2 cuentas Scale.

Estos pagos son mensuales, por lo que se estima un coste de 198\$ al mes en caso de que cada mes se quiera realizar una nueva extracción.
Haciendo la conversión a euros, a día 3 de julio de 2025, el coste sería de \textbf{167,92 euros al mes}. (1 USD = 0,85 EUR)

Si se decide pagar anualmente, se aplicaría un descuento del 10\% sobre cada cuenta, lo que supondría un coste de 179\$ al mes entre las dos cuentas y de 2148\$ anualmente.
Haciendo la conversión a euros, el coste sería de \textbf{1821,63 euros al año}. (1 USD = 0,85 EUR) 

Estos costes han sido calculados teniendo en cuenta que se aplicaría un descuento del 50\% por cuentas educativas.
Cabe destacar que estos cálculos están pensados para una futura escalabilidad del proyecto. Para el desarrollo del mismo se ha optado por utilizar cuentas gratuitas y un scraper de código abierto gratuito.

\subsubsection{Costes de alojamiento}

Para el alojamiento de la base de datos, se ha optado por utilizar un servidor en la nube de Microsoft Azure.
La opción elegida ha sido la opción serverless ya que es la más económica y permite alojar la base de datos correctamente.
En el modelo Serverless, se paga 5,69 USD al mes por 41,6 GB de almacenamiento, además de un coste adicional de 0,000159 USD por cada segundo de procesamiento de la CPU. En cambio, con un servidor dedicado, el coste mensual asciende a 396,10 USD por los mismos 41,6 GB de almacenamiento, incluyendo dos núcleos de CPU.
Respecto al alojamiento del cuadro de mando en PowerPages es necesaria una cuenta de Power BI Pro que tiene un coste de 13,10 euros al mes.
De esta forma se estima un coste mensual de entre 6 y 8 USD, lo que al cambio a euros sería aproximadamente de \textbf{5 a 7 euros al mes}. (1 USD = 0,85 EUR)

\subsection{Viabilidad legal}

\subsubsection{Protección de datos}

Respecto a la protección de datos, se ha procurado mantener en el anónimato los datos extraídos. 
No se trata de un proyecto con ánimo de lucro por lo que no se pretende obtener ningún beneficio económico de los datos extraídos.
En ningún momento del cuadro de mando se muestra información personal de los usuarios, únicamente se muestra la distinción entre género.

\subsubsection{Extracción de datos}

Si bien es cierto que la extracción de datos mediante scraper es un tema controvertido, en España, realizar web scraping es completamente legal.
En este proyecto, la información extraída es de acceso público y no se está vulnerando ningún derecho de propiedad intelectual.\cite{webscraping}

\subsubsection{Licencias de software}

Durante el desarrollo de este proyecto se han utilizado las siguientes herramientas con sus correspondientes licencias software:

\begin{table}[H]
	\centering
	\begin{tabularx}{\linewidth}{ p{0.4\columnwidth} p{0.3\columnwidth} p{0.3\columnwidth} }
		\toprule
		\textbf{Herramienta}    & \textbf{Versión}      &\textbf{Licencia}\\
        \toprule
		Visual Studio Code      & 1.101.2               & MIT \\
        LaTeX                   & 2023                  & LPPL \\
        Google Colab            & N/A                   & Propietaria \\
        Git                     & 2.47.1                & GPLv2 \\
        GitHub                  & N/A                   & Propietaria \\
        Python                  & 3.10.11               & PSF \\
        TensorFlow              & 2.18.0                & Apache 2.0 \\
        Keras                   & 3.8.0                 & Apache 2.0 \\
        Numpy                   & 2.0.2                 & BSD \\
        Scikit Learn            & 1.6.1                 & BSD \\
        Pysentimiento           & 0.7.3                 & MIT \\
        Emoji                   & 2.14.1                & BSD \\
        Pyodbc                  & 5.2.0                 & MIT \\
        Langdetect              & 1.0.9                 & Apache 2.0 \\
        Gender\_guesser         & 0.4.0                 & GPLv3 \\
        Deep\_translator        & 1.11.4                & MIT \\
        Matplotlib              & 3.10.0                & PSF \\
        Seaborn                 & 0.13.2                & BSD \\
        Pandas                  & 2.2.2                 & GPLv3 \\
        Transformers            & 4.53.0                & Apache 2.0 \\
        Apify\_client           & 1.10.0                & Apache 2.0 \\
        Requests                & 2.32.4                & Apache 2.0 \\
        SSMS                    & 21.2.5                & Propietaria \\
        PowerBI                 & 2.140.1577.0          & Propietaria \\
        Draw.io                 & 27.0.6                & Apache 2.0 \\
        Docker                  & 27.4.0                & Apache 2.0 \\
        SonarQube               & 24.0.2                & GPLv3 \\
        Google Maps Scraper     & 1.0                   & MIT \\
        8icons                  & 2025                  & Icons8 UML \\
        httpx                   & 0.28.1                & BSD \\
        Dotenv                  & 1.1.1                 & BSD \\
		\bottomrule
	\end{tabularx}
	\caption{Tabla de herramientas y licencias}
\end{table}

\subsubsection{Propiedad intelectual}

El proyecto se puede dividir en varios productos finales con diferentes licencias:

\begin{table}[H]
	\centering
	\begin{tabularx}{\linewidth}{ p{0.7\columnwidth} p{0.3\columnwidth} }
		\toprule
		\textbf{Recurso}             &\textbf{Licencia}\\
        \toprule
		Dataset                      & CC BY-NC-SA \\
        Cuadro de mando              & CC BY-NC-SA \\
        Código fuente                & CC BY-NC-SA \\
        Documentación                & CC BY-NC-SA \\
		\bottomrule
	\end{tabularx}
	\caption{Tabla de productos y licencias}
\end{table}

Esto implica que el dataset, cuadro de mando y documentación pueden ser utilizados por cualquier persona siempre que se mantenga la misma licencia y se reconozca al autor original, pero no se permite su uso comercial.
El código fuente, sin embargo, se puede utilizar, modificar y distribuir libremente siempre que se mantenga la misma licencia y se reconozca al autor original.