\capitulo{1}{Introducción}

\section{Marco del trabajo}

El turismo es uian actividad que conlleva que las personas se desplacen
de su lugar habitual de forma temporal para visitar otros lugares, ya sea por
ocio, negocios, cultura u otras razones. Esta actividad puede ser entendida
como una forma de consumo ya que los viajeros gastan dinero en bienes y
servicios durante su estancia en el destino turístico.\cite{turismo}

En España, el turismo es una de las principales fuentes de ingresos y
empleo. En 2023, la actividad turística supuso un 13,4\% del PIB del país.
Esto no es casualidad, sus playas, gastronomía, clima mediterráneo y la
riqueza artística y arquitectónica han logrado posicionarlo entre los tres
países más visitados en todo el mundo llegando a recibir alrededor de 85
millones de turistas en ese mismo año.\cite{turismoEspana2025}.

Debido a la importancia del turismo en la economía española, la Sociedad
Mercantil Estatal para la Gestión de la Innovación y las Tecnologías Turísticas (SEGITTUR) lleva varios años trabajando para contribuir al desarrollo,
modernización y mantenimento de la industria turística española a través
de la innovación tecnológica. Trata de mejorar la competividad, calidad
y sostenibilidad en los ámbitos medioambientales, económicos y sociales
relacionados con el turismo.\cite{segitturEjes}

Con el objetivo de mejorar la competividad de los destinos turísticos
y la calidad de vida de los viajeros, la Secretaría de Estado de Turismo
(SETUR) promueve el programa Destino Turístico Inteligente (DTI) el cual
es gestionado por SEGITTUR. Un DTI es un destino turístico innovador con
una accesibilidad global y que cuenta con una infraestructura tecnológica avanzada. Esta infraestructura garantiza el desarrollo sostenible del destino
y trata de mejorar la experiencia de los viajeros y su calidad de vida.\cite{segittur:dti}

\section{Contenido del trabajo}

El objetivo principal del trabajo es la creacón y publicación de un cuadro
de mando con el objetivo de que la provincia de Burgos alcance el distintivo
de Destino Turístico Inteligente. Para ello, se ha desarrollado un sistema
de extracción automatizado de reseñas de puntos de interés en la provincia
de Burgos. Con esa información se crea el cuadro de mando que permite
analizar y monitorizar distintas métricas relevantes como la distribución de
los puntos de interés por categorias. Además, se ha desarrollado una red
neuronal entrenada con un conjunto de datos formado por reseñas extraídas
de la misma forma con el objetivo de clasificar por categorías las reseñas
proporcionadas externamente.

\section{Estructura de la memoria}

La memoria del proyecto se estructura en siete capítulos:

\begin{itemize}
    \item \textbf{Capítulo 1: Introducción.} Presenta el marco del trabajo, el contenido y la estructura de la memoria.
    \item \textbf{Capítulo 2: Objetivos del proyecto.} Define los objetivos generales y específicos del proyecto.
    \item \textbf{Capítulo 3: Conceptos teóricos.} : Explica los conceptos teóricos necesarios para entender el proyecto.
    \item \textbf{Capítulo 4: Técnicas y herramientas.} Describe las técnicas y herramientas utilizadas en el desarrollo del proyecto.
    \item \textbf{Capítulo 5: Aspectos relevantes del desarrollo del proyecto.} Detalla el desarrollo del proyecto, incluyendo la extracción de reseñas, la creación del cuadro de mando y el entrenamiento de la red neuronal.
    \item \textbf{Capítulo 6: Trabajos relacionados.} Presenta una revisión de trabajos relacionados con el proyecto, incluyendo estudios previos y proyectos similares.
    \item \textbf{Capítulo 7: Conclusiones y líneas de trabajo futuras.} Resume las conclusiones del proyecto y propone posibles trabajos futuros.
\end{itemize}

Además, se incluten varios anexos que contienen información adicional relevante para el proyecto:

\begin{itemize}
    \item \textbf{Anexo A: Plan de proyecto software.} Contiene el plan de proyecto, incluyendo la planificación temporal y el desarrollo del mismo y la viabilidad económica y legal.
    \item \textbf{Anexo B: Especificación de requisitos.} Incluye los requisitos funcionales y no funcionales del proyecto, así como los casos de uso. 
    \item \textbf{Anexo C: Especificación de diseño.} Explica el diseño del sistema, incluyendo la arquitectura o las interfaces entre otras cosas.
    \item \textbf{Anexo D: Documentación técnica de programación.} Describe la parte técnica del proyecto como las explicaciones sobre el repositorio del código fuente.
    \item \textbf{Anexo E: Documentación de usuario.} Incluye de forma detallada todo lo que el usuario necesita saber y poseer para poder ejecutar y/o utilizar el proyecto principalmente en forma de manual de usuario.
    \item \textbf{Anexo F: Anexo de sostenibilización curricular.} Presenta una reflexión sobre los aspectos de sostenibilidad abordados en el proyecto.
\end{itemize}

\section{Materiales entregados}

Los materiales entregados al finalizar el proyecto son:

\begin{itemize}
    \item Varios scripts de Python utilizados para la extracción de reseñas.
    \item Un conjunto de datos con las reseñas extraídas utilizadas para entrenar la red neuronal.
    \item Un modelo de red neuronal entrenado para clasificar reseñas por categorías.   
    \item Un cuadro de mando interactivo creado con Power BI publicado en Power Pages.
    \item Una memoria del proyecto junto a sus anexos en formato PDF creado con \LaTeX.
    \item Un repositorio de GitHub con el código fuente del proyecto y la documentación.\cite{repositorio}
\end{itemize}
