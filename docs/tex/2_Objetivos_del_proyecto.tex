\capitulo{2}{Objetivos del proyecto}

Este proyecto tiene como finalidad desarrollar una solución tecnológica que permita analizar la reputación online de distintos destinos turísticos.

\section {Crear un cuadro de mando}
Crear un cuadro de mando destinado al análisis de la reputación online, aplicable a distintos destinos turísticos.
Este cuadro de mando contendrá información visual y estructurada sobre la percepción de los usuarios sobre dichos destinos turísticos.

\section {Sistema de análisis de sentimientos}
El análisis de sentimientos permite interpretar de forma automática la opinión de los usuarios sobre un destino turístico.
Se desarrollará un sistema de análisis de sentimientos que permita obtener información sobre la percepción de los usuarios.
Se utilizarán reseñas de usuarios para extraer dicha información sobre los destinos turísticos.

\section {Clasificar e interpretar los datos}
Hacer una clasficación e interpretación de los datos obtenidos: recursos, reseñas, etc. El objetivo de este proceso es obtener información relevante.
Para ello, se clasificarán las reseñas de los usuarios en positivas, negativas o neutras.
También se aplicara el mismo procedimiento para las palabras más frecuentes en las reseñas.
Esta tarea se realizará con PowerBI, una herramienta de análisis de datos.

\section {Automatizar la obtención de datos}
 
Para que el sistema de análisis de reputación online sea eficiente y aplicable a múltiples destinos turísticos, es fundamental automatizar el proceso de recopilación de datos. 
Esto permitirá que el sistema escale sin intervención manual constante, garantizando una actualización en tiempo real y una mayor precisión en la información analizada.

\section {Mostrar los resultados}
Mostrar dichos resultados a través de una página web creada con PowerPages con el fin de que los datos sean accesibles y comprensibles fácilmente por los usuarios.
Dicha plataforma permite la creación de páginas web de forma sencilla y rápida sin necesidad de escribir código.
