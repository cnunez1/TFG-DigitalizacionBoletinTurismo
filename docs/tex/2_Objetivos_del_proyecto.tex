\capitulo{2}{Objetivos del proyecto}

Este proyecto tiene como finalidad desarrollar una solución tecnológica que permita analizar la reputación online de distintos destinos turísticos 
mediante un cuadro de mando desde el que se puedan visualizar los datos obtenidos relativos a las reseñas de forma sencilla y comprensible.
Para ello, se desarrollará un sistema de análisis de reputación online que permita obtener información relevante sobre la percepción de los usuarios sobre un destino turístico.
Realizar esta tarea conlleva los siguientes pasos:

\begin{itemize}
    \item Automatizar la obtención de datos.
    \item Clasificar e interpretar los datos.
    \item Sistema de análisis de sentimientos.
    \item Crear un cuadro de mando.
    \item Publicar los resultados.
\end{itemize}

\section {Automatizar la obtención de datos}
 
Para que el sistema de análisis de reputación online sea eficiente y aplicable a múltiples destinos turísticos, es fundamental automatizar el proceso de recopilación de datos.
Se desarrollará un sistema de extracción de datos automatizado que permita obtener información relevante de distintas fuentes, como reseñas de usuarios, puntuaciones, etc. 
Estos datos están publicados en Internet de forma altruista por diferentes personas que buscan aconsejar a otros usuarios.
Esto permitirá que el sistema escale sin intervención manual constante, garantizando una actualización en tiempo real y una mayor precisión en la información analizada.

\section {Clasificar e interpretar los datos}
Hacer una clasficación e interpretación de los datos obtenidos previamente: recursos, reseñas, etc. El objetivo de este proceso es obtener información relevante.
Para ello, se clasificarán las reseñas de los usuarios en positivas, negativas o neutras.
Además, se analizarán las palabras más frecuentes para identificar temas recurrentes en las reseñas. 
Esta clasificación será clave para el siguiente paso del análisis de sentimientos.

\section {Sistema de análisis de sentimientos}
El análisis de sentimientos permite interpretar de forma automática la opinión de los usuarios sobre un destino turístico.
Se desarrollará un sistema de análisis de sentimientos que permita obtener información sobre la percepción de los usuarios.
Se utilizarán reseñas de usuarios para extraer dicha información sobre los destinos turísticos.
El proceso ETL (Extract, Transform, Load) es fundamental para preparar las reseñas de usuarios para el análisis de sentimientos.
Para que el sistema de análisis de sentimientos funcione correctamente, los datos deben estar estructurados, como puntuaciones y palabras clave, que faciliten el análisis. 
Sin embargo, muchas reseñas son textos no estructurados. 
En estos casos, el sub-sistema de Text Mining se encarga de procesar y extraer información útil del texto, como palabras clave y sentimientos, utilizando técnicas de procesamiento de lenguaje natural. 
Esto convierte los datos no organizados en información estructurada que luego se puede analizar con el objetivo de evaluar la percepción de los usuarios sobre los destinos turísticos.

\section {Crear un cuadro de mando}
Crear un cuadro de mando interactivo destinado al análisis de la reputación online, aplicable a distintos destinos turísticos.
Este cuadro de mando contendrá información visual y estructurada sobre la percepción de los usuarios sobre dichos destinos turísticos.
Esta tarea se realizará con PowerBI, una herramienta de análisis de datos que permite crear gráficos e informes visuales.
PowerBI permite crear cuadros de mando interactivos y visuales para visualizar los datos de forma clara y concisa. 
Además, permite usar DAX, un lenguaje de fórmulas que permite realizar cálculos para obtener medidas útiles en el análisis de datos.

\section {Publicar los resultados}
Finalmente, para que los resultados sean accesibles y comprensibles para todos, se publicarán en una página web creada con PowerPages en la que se alojará el cuadro de mando creado con PowerBI previamente. 
Esto se hace con el fin de que los datos sean accesibles y comprensibles fácilmente por los usuarios.
Dicha plataforma permite la creación de páginas web de forma sencilla y rápida sin necesidad de escribir código. 
Se puede alojar el cuadro de mando de PowerBI simplemente copiando su URL o su código en HTML.
Dado que el objetivo es únicamente mostrar información no es necesario implementar una sistema de cuentas.
En relación con esto, es suficiente con crear una cuenta de administrador que configure las consultas necesarias.