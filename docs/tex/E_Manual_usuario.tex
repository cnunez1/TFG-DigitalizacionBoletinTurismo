\apendice{Documentación de usuario}

\section{Introducción}

En este apéndice se presenta la documentación de usuario necesaria para que el usuario final pueda utilizar el sistema desarrollado en este proyecto.
Este apéndice está enfocado al usuario final, que en principio no tiene conocimientos técnicos sobre el sistema y cuyo objetivo es usar el cuadro de mando final.

\section{Requisitos de usuarios}

El usuario final debe contar con varios requisitos para poder utilizar el cuadro de mando:
\begin{itemize}
    \item Tener acceso a un navegador web moderno con conexión a Internet.
    \item Tener acceso a una cuenta de Power BI y su aplicación descargada para poder acceder al cuadro de mando.
\end{itemize}

\section{Instalación}

La 'instalación' del cuadro de mando es un proceso muy simple. Simplemente hay que abrir el fichero \texttt{dashboardFinal.pbix} con la aplicación de Power BI Desktop. Una vez abierto, el usuario podrá ver el cuadro de mando y navegar por él.

\section{Manual del usuario}

El cuadro de mando está dividido en varias páginas. 
El usuario puede navegar de dos formas diferentes a través de dos componentes diferentes.

\subsection{Navegación por pestañas}

PowerBI cuenta con un menú inferior con las pestañas del cuadro de mando. Simplemente haciendo clic sobre ellas se puede navegar entre ellas.

\imagen{pestanas}{Pestañas de navegación}{1}

\subsection{Navegación por botones (sidebar)}

También se ha implementado de forma manual una barra lateral con iconos para navegar entre las difernetes pestañas. 
Para usarlos, el usuario debe pulsar el botón CONTROL a la vez que hace clic en el icono deseado.
Se muestra a continuación la barra lateral de forma apaisada. (En el cuadro de mando se encuentra a la izquierda en vertical).
\imagen{sidebar}{Barra lateral}{1}

\subsection{Filtros}

El cuadro de mando cuenta con dos filtros principales basados en los tipos de recursos y municipios. 
Estos filtros están disponibles en todas las pestañas.
\imagen{filtros}{Filtros de recursos y municipios}{0.5}

\subsection{Métricas}

Todas las gráficas del cuadro de mando son interactivas e intuitivas.
El usuario puede clicar en ellas para filtrar de la misma forma que con los filtros principales.
Las métricas principales del cuadro de mando son el TORI, el número de reseñas y la polaridad.
\imagen{metricas}{Métricas principales}{1}

\subsection{Borrado de filtros}

Se cuenta con un botón para borrar los filtros aplicados en el cuadro de mando.
Este botón se encuentra en la parte superior derecha y se acciona pulsando CONTROL a la vez que se hace clic sobre él.
\imagen{borrarFiltros}{Botón de borrado de filtros}{0.1}

\subsection{Ayuda}

Finalmente, cada pestaña cuenta con una sección de ayuda a la derecha que explica las funcionalidades desarrolladas en cada página.

\subsection{Páginas del cuadro de mando}
A continuación se muestran dos ejemplos de páginas del cuadro de mando en el que se ven los aspectos anteriores.
\imagen{recursos}{Página de recursos}{1}
\imagen{evolucionTemporal}{Página de evolución temporal}{1}