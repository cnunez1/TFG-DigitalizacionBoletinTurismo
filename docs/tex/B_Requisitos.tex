\apendice{Especificación de Requisitos}

\section{Introducción}

En este anexo se especifican los requisitos funcionales y no funcionales del sistema, así como los casos de uso que describen las interacciones entre los actores y el sistema.
A continuación se listan los objetivos de los que nacen los requisitos funcionales y no funcionales junto a los casos de uso que los implementan.

\section{Objetivos generales}

Este proyecto cuenta con varios objetivos generales detallados a continuación:

\begin{itemize}
	\item Automatizar la extracción de recursos y reseñas de Google Maps.
	\item Desarrollar una red neuronal que permita clasificar los recursos según su tipo.
	\item Analizar los datos obtenidos con herramientas de inteligencia de negocio para obtener métricas y visualizaciones útiles.
	\item Publicar los resultados en un sitio web para que sean accesibles al público.
\end{itemize}

\section{Catálogo de requisitos}

\subsection{Requisitos funcionales}

\begin{itemize}
	\item \textbf{RF-1 Extraer los datos:} El sistema debe permitir la extracción de datos de Google Maps.
		\begin{itemize}
			\item \textbf{RF-1.1 Extraer recursos:} El sistema debe extraer información de recursos como restaurantes, hoteles, etc.
			\item \textbf{RF-1.2 Extraer reseñas:} El sistema debe extraer reseñas de los recursos obtenidos.
			\item \textbf{RF-1.3 Configurar parámetros de extracción:} El sistema debe permitir la configuración de parámetros como número de reseñas, idioma, etc.
		\end{itemize}
	\item \textbf{RF-2 Clasificar con aprendizaje automático:} El sistema debe clasificar los recursos extraídos utilizando una red neuronal.
		\begin{itemize}
			\item \textbf{RF-2.1 Entrenar la red neuronal:} El sistema debe permitir el entrenamiento de la red neuronal con los datos extraídos.
			\item \textbf{RF-2.2 Clasificar recursos:} El sistema debe clasificar los recursos en diferentes categorías (por ejemplo, restaurantes, hoteles, etc.).
			\item \textbf{RF-2.3 Evaluar la precisión del modelo:} El sistema debe evaluar la precisión del modelo de clasificación utilizando métricas adecuadas.
			\item \textbf{RF-2.4 Ajustar el modelo:} El sistema debe permitir ajustar el modelo de clasificación para mejorar su precisión.
			\item \textbf{RF-2.5 Guardar el modelo:} El sistema debe guardar el modelo entrenado para su uso posterior.
			\item \textbf{RF-2.6 Cargar el modelo:} El sistema debe permitir cargar un modelo previamente entrenado para su uso.
			\item \textbf{RF-2.7 Predecir tipos de recursos:} El sistema debe permitir predecir los tipos de nuevos recursos utilizando el modelo entrenado.
		\end{itemize}
	\item \textbf{RF-3 Analizar los resultados:} El sistema debe generar informes y visualizaciones a partir de los datos analizados.
		\begin{itemize}
			\item \textbf{RF-3.1 Visualizar el TORI:} El sistema debe permitir visualizar el TORI de los recursos.
			\item \textbf{RF-3.2 Visualizar el número de reseñas:} El sistema debe permitir visualizar el número de reseñas por recurso.
			\item \textbf{RF-3.3 Visualizar la puntuación media:} El sistema debe permitir visualizar la puntuación media de los recursos.
			\item \textbf{RF-3.4 Visualizar la distribución de categorías:} El sistema debe permitir visualizar la distribución de recursos por categoría.
			\item \textbf{RF-3.5 Visualizar la evolución temporal:} El sistema debe permitir visualizar la evolución de los recursos a lo largo del tiempo.
			\item \textbf{RF-3.6 Visualizar el mapa de recursos:} El sistema debe permitir visualizar un mapa con la ubicación de los recursos.
		\end{itemize}
	\item \textbf{RF-4 Publicar los resultados:} El sistema debe publicar los resultados en un sitio web accesible al público.
\end{itemize}

\subsubsection{Requisitos no funcionales}

\begin{itemize}
	\item \textbf{RNF-1 Disponibilidad:} El sistema debe estar disponible 24/7 para la extracción y análisis de datos.
	\item \textbf{RNF-2 Rendimiento:} El sistema debe ser capaz de procesar grandes volúmenes de datos en un tiempo razonable.
	\item \textbf{RNF-3 Usabilidad:} El sistema debe ser fácil de usar para los usuarios finales.
	\item \textbf{RNF-4 Seguridad:} El sistema debe garantizar la seguridad de los datos extraídos y analizados.
	\item \textbf{RNF-5 Portabilidad:} El sistema debe ser portablec por tanto, debe poder ejecutarse en diferentes entornos y máquinas.
\end{itemize}

\section{Especificación de requisitos}

\imagen{casos_de_uso}{Diagrama de casos de uso}{1}

\begin{table}[p]
	\centering
	\begin{tabularx}{\linewidth}{ p{0.21\columnwidth} p{0.71\columnwidth} }
		\toprule
		\textbf{CU-1}    & \textbf{Extraer los datos}\\
		\toprule
		\textbf{Versión}              & 1.0    \\
		\textbf{Autor}                & Christian Andrés Núñez Duque \\
		\textbf{Requisitos asociados} & RF-1, RF-1.1, RF-1.2, RF-1.3 \\
		\textbf{Descripción}          & Se deben extraer las reseñas y recursos y poder configurar su extracción \\
		\textbf{Precondición}         & Poseer una API KEY de Apify \\
		\textbf{Acciones}             &
		\begin{enumerate}
			\def\labelenumi{\arabic{enumi}.}
			\tightlist
			\item Determinar la API KEY de Apify en el fichero .env
			\item Seleccionar los recursos a extraer mediante placeID
			\item Seleccionar los parámetros de extracción (número de reseñas, idioma, etc.)
			\item Ejecutar el script de extracción
		\end{enumerate}\\
		\textbf{Postcondición}        & Fichero .csv con los datos extraídos \\
		\textbf{Excepciones}          & Caída de Apify \\
		\textbf{Importancia}          & Alta \\
		\bottomrule
	\end{tabularx}
	\caption{CU-1 Extraer los datos}
\end{table}

\begin{table}[p]
	\centering
	\begin{tabularx}{\linewidth}{ p{0.21\columnwidth} p{0.71\columnwidth} }
		\toprule
		\textbf{CU-2}    & \textbf{Extraer los recursos}\\
		\toprule
		\textbf{Versión}              & 1.0    \\
		\textbf{Autor}                & Christian Andrés Núñez Duque \\
		\textbf{Requisitos asociados} & RF-1.1 \\
		\textbf{Descripción}          & Se deben extraer los recursos de forma sencilla \\
		\textbf{Precondición}         & Poseer una API KEY de Apify \\
		\textbf{Acciones}             &
		\begin{enumerate}
			\def\labelenumi{\arabic{enumi}.}
			\tightlist
			\item Determinar la API KEY de Apify en el fichero .env
			\item Seleccionar los recursos a extraer mediante placeID
			\item Ejecutar el script de extracción
		\end{enumerate}\\
		\textbf{Postcondición}        & Fichero .csv con los recursos extraídos \\
		\textbf{Excepciones}          & Caída de Apify \\
		\textbf{Importancia}          & Alta \\
		\bottomrule
	\end{tabularx}
	\caption{CU-2 Extraer los recursos}
\end{table}

\begin{table}[p]
	\centering
	\begin{tabularx}{\linewidth}{ p{0.21\columnwidth} p{0.71\columnwidth} }
		\toprule
		\textbf{CU-3}    & \textbf{Extraer las reseñas}\\
		\toprule
		\textbf{Versión}              & 1.0    \\
		\textbf{Autor}                & Christian Andrés Núñez Duque \\
		\textbf{Requisitos asociados} & RF-1.2 \\
		\textbf{Descripción}          & Se deben extraer las reseñas de forma sencilla \\
		\textbf{Precondición}         & Poseer una API KEY de Apify \\
		\textbf{Acciones}             &
		\begin{enumerate}
			\def\labelenumi{\arabic{enumi}.}
			\tightlist
			\item Determinar la API KEY de Apify en el fichero .env
			\item Seleccionar los recursos a extraer mediante placeID
			\item Seleccionar los parámetros de extracción (número de reseñas, idioma, etc.)
			\item Ejecutar el script de extracción
		\end{enumerate}\\
		\textbf{Postcondición}        & Fichero .csv con las reseñas extraídas \\
		\textbf{Excepciones}          & Caída de Apify \\
		\textbf{Importancia}          & Alta \\
		\bottomrule
	\end{tabularx}
	\caption{CU-3 Extraer las reseñas}
\end{table}

\begin{table}[p]
	\centering
	\begin{tabularx}{\linewidth}{ p{0.21\columnwidth} p{0.71\columnwidth} }
		\toprule
		\textbf{CU-4}    & \textbf{Configurar parámetros de extracción}\\
		\toprule
		\textbf{Versión}              & 1.0    \\
		\textbf{Autor}                & Christian Andrés Núñez Duque \\
		\textbf{Requisitos asociados} & RF-1.3 \\
		\textbf{Descripción}          & El usuario debe poder configurar cómodamente los parámetros de la extracción de datos \\
		\textbf{Precondición}         & Poseer una API KEY de Apify \\
		\textbf{Acciones}             &
		\begin{enumerate}
			\def\labelenumi{\arabic{enumi}.}
			\tightlist
			\item Determinar la API KEY de Apify en el fichero .env
			\item Seleccionar los recursos a extraer mediante placeID
			\item Seleccionar los parámetros de extracción (número de reseñas, idioma, etc.)
			\item Ejecutar el script de extracción
		\end{enumerate}\\
		\textbf{Postcondición}        & Fichero .csv con los datos extraídos según los parámetros seleccionados \\
		\textbf{Excepciones}          & Caída de Apify \\
		\textbf{Importancia}          & Alta \\
		\bottomrule
	\end{tabularx}
	\caption{CU-4 Configurar parámetros de extracción}
\end{table}

\begin{table}[p]
	\centering
	\begin{tabularx}{\linewidth}{ p{0.21\columnwidth} p{0.71\columnwidth} }
		\toprule
		\textbf{CU-5}    & \textbf{Clasificar con aprendizaje automático}\\
		\toprule
		\textbf{Versión}              & 1.0    \\
		\textbf{Autor}                & Christian Andrés Núñez Duque \\
		\textbf{Requisitos asociados} & RF-2, RF-2.1, RF-2.2, RF-2.3, RF-2.4, RF-2.5, RF-2.6, RF-2.7 \\
		\textbf{Descripción}          & El usuario debe poder utilizar la red neuronal para predecir tipos de recursos \\
		\textbf{Precondición}         & Poseer el script de la red neuronal y el conjunto de datos \\
		\textbf{Acciones}             &
		\begin{enumerate}
			\def\labelenumi{\arabic{enumi}.}
			\tightlist
			\item Cargar el script/notebook en Google Colab
			\item Seleccionar GPU como entorno de ejecución
			\item Cargar el conjunto de datos de entrenamiento
			\item Ejecutar el script/notebook para entrenar la red neuronal
			\item Cargar el conjunto de recursos a clasificar
			\item Ejecutar el script/notebook para clasificar los recursos
		\end{enumerate}\\
		\textbf{Postcondición}        & Fichero .sql con los recursos clasificados \\
		\textbf{Excepciones}          & Caída de Google Colab \\ & Caída de Azure \\
		\textbf{Importancia}          & Alta \\
		\bottomrule
	\end{tabularx}
	\caption{CU-5 Clasificar con aprendizaje automático}
\end{table}

\begin{table}[p]
	\centering
	\begin{tabularx}{\linewidth}{ p{0.21\columnwidth} p{0.71\columnwidth} }
		\toprule
		\textbf{CU-6}    & \textbf{Entrenar la red neuronal}\\
		\toprule
		\textbf{Versión}              & 1.0    \\
		\textbf{Autor}                & Christian Andrés Núñez Duque \\
		\textbf{Requisitos asociados} & RF-2.1 \\
		\textbf{Descripción}          & El usuario debe poder reentrenar la red neuronal en cualquier momento \\
		\textbf{Precondición}         & Poseer el script de la red neuronal y el conjunto de datos \\
		\textbf{Acciones}             &
		\begin{enumerate}
			\def\labelenumi{\arabic{enumi}.}
			\tightlist
			\item Cargar el script/notebook en Google Colab
			\item Seleccionar GPU como entorno de ejecución
			\item Cargar el conjunto de datos de entrenamiento
			\item Ejecutar el script/notebook para entrenar la red neuronal
		\end{enumerate}\\
		\textbf{Postcondición}        & Ficheros .keras y .pkl con el modelo entrenado \\
		\textbf{Excepciones}          & Caída de Google Colab \\ & Caída de Azure \\
		\textbf{Importancia}          & Media \\
		\bottomrule
	\end{tabularx}
	\caption{CU-6 Entrenar la red neuronal}
\end{table}

\begin{table}[p]
	\centering
	\begin{tabularx}{\linewidth}{ p{0.21\columnwidth} p{0.71\columnwidth} }
		\toprule
		\textbf{CU-7}    & \textbf{Clasificar recursos}\\
		\toprule
		\textbf{Versión}              & 1.0    \\
		\textbf{Autor}                & Christian Andrés Núñez Duque \\
		\textbf{Requisitos asociados} & RF-2.2 \\
		\textbf{Descripción}          & El usuario debe poder clasificar recursos a partir del modelo \\
		\textbf{Precondición}         & Poseer los ficheros del modelo y de los recursos a clasificar \\
		\textbf{Acciones}             &
		\begin{enumerate}
			\def\labelenumi{\arabic{enumi}.}
			\tightlist
			\item Cargar el script/notebook en Google Colab
			\item Seleccionar GPU como entorno de ejecución
			\item Cargar los ficheros del modelo (si no están ya cargados) y los recursos a clasificar
			\item Ejecutar el script/notebook para clasificar recursos
		\end{enumerate}\\
		\textbf{Postcondición}        & Fichero .sql con los recursos clasificados \\
		\textbf{Excepciones}          & Caída de Google Colab \\ & Caída de Azure \\
		\textbf{Importancia}          & Alta \\
		\bottomrule
	\end{tabularx}
	\caption{CU-7 Clasificar recursos}
\end{table}

\begin{table}[p]
	\centering
	\begin{tabularx}{\linewidth}{ p{0.21\columnwidth} p{0.71\columnwidth} }
		\toprule
		\textbf{CU-8}    & \textbf{Evaluar la precisión del modelo}\\
		\toprule
		\textbf{Versión}              & 1.0    \\
		\textbf{Autor}                & Christian Andrés Núñez Duque \\
		\textbf{Requisitos asociados} & RF-2.3 \\
		\textbf{Descripción}          & El usuario debe poder ver el informe de clasificación del modelo y la matriz de confusión \\
		\textbf{Precondición}         & Poseer el script de la red neuronal y el conjunto de datos \\
		\textbf{Acciones}             &
		\begin{enumerate}
			\def\labelenumi{\arabic{enumi}.}
			\tightlist
			\item Cargar el script/notebook en Google Colab
			\item Seleccionar GPU como entorno de ejecución
			\item Cargar el conjunto de datos de entrenamiento
			\item Ejecutar el script/notebook para entrenar la red neuronal
		\end{enumerate}\\
		\textbf{Postcondición}        & Informe de clasificación y matriz de confusión en la terminal \\
		\textbf{Excepciones}          & Caída de Google Colab \\ & Caída de Azure \\
		\textbf{Importancia}          & Media \\
		\bottomrule
	\end{tabularx}
	\caption{CU-8 Evaluar la precisión del modelo}
\end{table}

\begin{table}[p]
	\centering
	\begin{tabularx}{\linewidth}{ p{0.21\columnwidth} p{0.71\columnwidth} }
		\toprule
		\textbf{CU-9}    & \textbf{Ajustar el modelo}\\
		\toprule
		\textbf{Versión}              & 1.0    \\
		\textbf{Autor}                & Christian Andrés Núñez Duque \\
		\textbf{Requisitos asociados} & RF-2.4 \\
		\textbf{Descripción}          & El usuario debe poder ajustar los hiperparámetros del modelo \\
		\textbf{Precondición}         & Poseer el script de la red neuronal y el conjunto de datos \\
		\textbf{Acciones}             &
		\begin{enumerate}
			\def\labelenumi{\arabic{enumi}.}
			\tightlist
			\item Cargar el script/notebook en Google Colab
			\item Seleccionar GPU como entorno de ejecución
			\item Cargar el conjunto de datos de entrenamiento
			\item Ajustar los hiperparámetros del modelo (número de capas, neuronas, tasa de aprendizaje, etc.)
			\item Ejecutar el script/notebook para entrenar la red neuronal
		\end{enumerate}\\
		\textbf{Postcondición}        & Informe de clasificación y matriz de confusión en la terminal \\
		\textbf{Excepciones}          & Caída de Google Colab \\ & Caída de Azure \\
		\textbf{Importancia}          & Media \\
		\bottomrule
	\end{tabularx}
	\caption{CU-9 Evaluar la precisión del modelo}
\end{table}

\begin{table}[p]
	\centering
	\begin{tabularx}{\linewidth}{ p{0.21\columnwidth} p{0.71\columnwidth} }
		\toprule
		\textbf{CU-10}    & \textbf{Guardar el modelo}\\
		\toprule
		\textbf{Versión}              & 1.0    \\
		\textbf{Autor}                & Christian Andrés Núñez Duque \\
		\textbf{Requisitos asociados} & RF-2.5 \\
		\textbf{Descripción}          & El usuario debe poder guardar el modelo para su posterior uso \\
		\textbf{Precondición}         & Poseer el script de la red neuronal y el conjunto de datos \\
		\textbf{Acciones}             &
		\begin{enumerate}
			\def\labelenumi{\arabic{enumi}.}
			\tightlist
			\item Cargar el script/notebook en Google Colab
			\item Seleccionar GPU como entorno de ejecución
			\item Cargar el conjunto de datos de entrenamiento
			\item Ejecutar el script/notebook para entrenar la red neuronal
		\end{enumerate}\\
		\textbf{Postcondición}        & Ficheros .keras y .pkl con el modelo entrenado \\
		\textbf{Excepciones}          & Caída de Google Colab \\ & Caída de Azure \\
		\textbf{Importancia}          & Baja \\
		\bottomrule
	\end{tabularx}
	\caption{CU-10 Guardar el modelo}
\end{table}

\begin{table}[p]
	\centering
	\begin{tabularx}{\linewidth}{ p{0.21\columnwidth} p{0.71\columnwidth} }
		\toprule
		\textbf{CU-11}    & \textbf{Cargar el modelo}\\
		\toprule
		\textbf{Versión}              & 1.0    \\
		\textbf{Autor}                & Christian Andrés Núñez Duque \\
		\textbf{Requisitos asociados} & RF-2.6 \\
		\textbf{Descripción}          & El usuario debe poder cargar el modelo para la predicción de tipos de recurso \\
		\textbf{Precondición}         & Poseer los ficheros del modelo \\
		\textbf{Acciones}             &
		\begin{enumerate}
			\def\labelenumi{\arabic{enumi}.}
			\tightlist
			\item Cargar el script/notebook en Google Colab
			\item Seleccionar GPU como entorno de ejecución
			\item Cargar los ficheros del modelo (si no están ya cargados) y los recursos a clasificar
		\end{enumerate}\\
		\textbf{Postcondición}        & Mensaje en la terminal de "modelo cargado" \\
		\textbf{Excepciones}          & Caída de Google Colab \\ & Caída de Azure \\
		\textbf{Importancia}          & Baja \\
		\bottomrule
	\end{tabularx}
	\caption{CU-11 Cargar el modelo}
\end{table}

\begin{table}[p]
	\centering
	\begin{tabularx}{\linewidth}{ p{0.21\columnwidth} p{0.71\columnwidth} }
		\toprule
		\textbf{CU-12}    & \textbf{Predecir tipos de recurso}\\
		\toprule
		\textbf{Versión}              & 1.0    \\
		\textbf{Autor}                & Christian Andrés Núñez Duque \\
		\textbf{Requisitos asociados} & RF-2.7 \\
		\textbf{Descripción}          & El usuario debe poder predecir tipos de recursos \\
		\textbf{Precondición}         & Poseer los ficheros del modelo \\
		\textbf{Acciones}             &
		\begin{enumerate}
			\def\labelenumi{\arabic{enumi}.}
			\tightlist
			\item Cargar el script/notebook en Google Colab
			\item Seleccionar GPU como entorno de ejecución
			\item Cargar los ficheros del modelo (si no están ya cargados) y los recursos a clasificar
			\item Ejecutar el script/notebook para clasificar recursos
		\end{enumerate}\\
		\textbf{Postcondición}        & Fichero .sql con los recursos clasificados \\
		\textbf{Excepciones}          & Caída de Google Colab \\ & Caída de Azure \\
		\textbf{Importancia}          & Alta \\
		\bottomrule
	\end{tabularx}
	\caption{CU-12 Predecir tipos de recurso}
\end{table}

\begin{table}[p]
	\centering
	\begin{tabularx}{\linewidth}{ p{0.21\columnwidth} p{0.71\columnwidth} }
		\toprule
		\textbf{CU-13}    & \textbf{Analizar los resultados}\\
		\toprule
		\textbf{Versión}              & 1.0    \\
		\textbf{Autor}                & Christian Andrés Núñez Duque \\
		\textbf{Requisitos asociados} & RF-3, RF-3.1, RF-3.2, RF-3.3, RF-3.4, RF-3.5, RF-3.6 \\
		\textbf{Descripción}          & El usuario debe poder visualizar el análisis de resultados \\
		\textbf{Precondición}         & Poseer el fichero .pbix de PowerBI con el cuadro de mando y las credenciales de acceso a la base de datos \\
		\textbf{Acciones}             &
		\begin{enumerate}
			\def\labelenumi{\arabic{enumi}.}
			\tightlist
			\item Ejecutar el fichero .pbix de PowerBI
			\item Escribir las credenciales de acceso a la base de datos
			\item Visualizar el cuadro de mando con los datos analizados
		\end{enumerate}\\
		\textbf{Postcondición}        & Ver métricas y gráficas \\
		\textbf{Excepciones}          & Caída de Azure \\
		\textbf{Importancia}          & Alta \\
		\bottomrule
	\end{tabularx}
	\caption{CU-13 Analizar los resultados}
\end{table}

\begin{table}[p]
	\centering
	\begin{tabularx}{\linewidth}{ p{0.21\columnwidth} p{0.71\columnwidth} }
		\toprule
		\textbf{CU-14}    & \textbf{Visualizar el TORI}\\
		\toprule
		\textbf{Versión}              & 1.0    \\
		\textbf{Autor}                & Christian Andrés Núñez Duque \\
		\textbf{Requisitos asociados} & RF-3.1 \\
		\textbf{Descripción}          & El usuario debe poder visualizar el TORI \\
		\textbf{Precondición}         & Poseer el fichero .pbix de PowerBI con el cuadro de mando y las credenciales de acceso a la base de datos \\
		\textbf{Acciones}             &
		\begin{enumerate}
			\def\labelenumi{\arabic{enumi}.}
			\tightlist
			\item Ejecutar el fichero .pbix de PowerBI
			\item Escribir las credenciales de acceso a la base de datos
			\item Visualizar el TORI de los recursos
		\end{enumerate}\\
		\textbf{Postcondición}        & Ver el TORI de los recursos \\
		\textbf{Excepciones}          & Caída de Azure \\
		\textbf{Importancia}          & Alta \\
		\bottomrule
	\end{tabularx}
	\caption{CU-14 Visualizar el TORI}
\end{table}

\begin{table}[p]
	\centering
	\begin{tabularx}{\linewidth}{ p{0.21\columnwidth} p{0.71\columnwidth} }
		\toprule
		\textbf{CU-15}    & \textbf{Visualizar el número de reseñas}\\
		\toprule
		\textbf{Versión}              & 1.0    \\
		\textbf{Autor}                & Christian Andrés Núñez Duque \\
		\textbf{Requisitos asociados} & RF-3.2 \\
		\textbf{Descripción}          & El usuario debe poder visualizar el número de reseñas \\
		\textbf{Precondición}         & Poseer el fichero .pbix de PowerBI con el cuadro de mando y las credenciales de acceso a la base de datos \\
		\textbf{Acciones}             &
		\begin{enumerate}
			\def\labelenumi{\arabic{enumi}.}
			\tightlist
			\item Ejecutar el fichero .pbix de PowerBI
			\item Escribir las credenciales de acceso a la base de datos
			\item Visualizar el número de reseñas de los recursos
		\end{enumerate}\\
		\textbf{Postcondición}        & Ver el número de reseñas de los recursos \\
		\textbf{Excepciones}          & Caída de Azure \\
		\textbf{Importancia}          & Media \\
		\bottomrule
	\end{tabularx}
	\caption{CU-15 Visualizar el número de reseñas}
\end{table}

\begin{table}[p]
	\centering
	\begin{tabularx}{\linewidth}{ p{0.21\columnwidth} p{0.71\columnwidth} }
		\toprule
		\textbf{CU-16}    & \textbf{Visualizar la puntuación media}\\
		\toprule
		\textbf{Versión}              & 1.0    \\
		\textbf{Autor}                & Christian Andrés Núñez Duque \\
		\textbf{Requisitos asociados} & RF-3.3 \\
		\textbf{Descripción}          & El usuario debe poder visualizar la puntuación media de los recursos \\
		\textbf{Precondición}         & Poseer el fichero .pbix de PowerBI con el cuadro de mando y las credenciales de acceso a la base de datos \\
		\textbf{Acciones}             &
		\begin{enumerate}
			\def\labelenumi{\arabic{enumi}.}
			\tightlist
			\item Ejecutar el fichero .pbix de PowerBI
			\item Escribir las credenciales de acceso a la base de datos
			\item Visualizar la puntuación media de los recursos
		\end{enumerate}\\
		\textbf{Postcondición}        & Ver la puntuación media de los recursos \\
		\textbf{Excepciones}          & Caída de Azure \\
		\textbf{Importancia}          & Media \\
		\bottomrule
	\end{tabularx}
	\caption{CU-16 Visualizar la puntuación media}
\end{table}

\begin{table}[p]
	\centering
	\begin{tabularx}{\linewidth}{ p{0.21\columnwidth} p{0.71\columnwidth} }
		\toprule
		\textbf{CU-17}    & \textbf{Visualizar la evolución temporal}\\
		\toprule
		\textbf{Versión}              & 1.0    \\
		\textbf{Autor}                & Christian Andrés Núñez Duque \\
		\textbf{Requisitos asociados} & RF-3.5 \\
		\textbf{Descripción}          & El usuario debe poder visualizar la evolución temporal de las reseñas \\
		\textbf{Precondición}         & Poseer el fichero .pbix de PowerBI con el cuadro de mando y las credenciales de acceso a la base de datos \\
		\textbf{Acciones}             &
		\begin{enumerate}
			\def\labelenumi{\arabic{enumi}.}
			\tightlist
			\item Ejecutar el fichero .pbix de PowerBI
			\item Escribir las credenciales de acceso a la base de datos
			\item Visualizar la evolución temporal de las reseñas
		\end{enumerate}\\
		\textbf{Postcondición}        & Ver la evolución temporal de las reseñas \\
		\textbf{Excepciones}          & Caída de Azure \\
		\textbf{Importancia}          & Media \\
		\bottomrule
	\end{tabularx}
	\caption{CU-17 Visualizar la evolución temporal}
\end{table}

\begin{table}[p]
	\centering
	\begin{tabularx}{\linewidth}{ p{0.21\columnwidth} p{0.71\columnwidth} }
		\toprule
		\textbf{CU-18}    & \textbf{Visualizar el mapa de recursos}\\
		\toprule
		\textbf{Versión}              & 1.0    \\
		\textbf{Autor}                & Christian Andrés Núñez Duque \\
		\textbf{Requisitos asociados} & RF-3.6 \\
		\textbf{Descripción}          & El usuario debe poder visualizar el mapa de recursos \\
		\textbf{Precondición}         & Poseer el fichero .pbix de PowerBI con el cuadro de mando y las credenciales de acceso a la base de datos \\
		\textbf{Acciones}             &
		\begin{enumerate}
			\def\labelenumi{\arabic{enumi}.}
			\tightlist
			\item Ejecutar el fichero .pbix de PowerBI
			\item Escribir las credenciales de acceso a la base de datos
			\item Visualizar el mapa de recursos
		\end{enumerate}\\
		\textbf{Postcondición}        & Ver el mapa de recursos \\
		\textbf{Excepciones}          & Caída de Azure \\
		\textbf{Importancia}          & Media \\
		\bottomrule
	\end{tabularx}
	\caption{CU-18 Visualizar el mapa de recursos}
\end{table}

\begin{table}[p]
	\centering
	\begin{tabularx}{\linewidth}{ p{0.21\columnwidth} p{0.71\columnwidth} }
		\toprule
		\textbf{CU-19}    & \textbf{Publicar los resultados}\\
		\toprule
		\textbf{Versión}              & 1.0    \\
		\textbf{Autor}                & Christian Andrés Núñez Duque \\
		\textbf{Requisitos asociados} & RF-3 \\
		\textbf{Descripción}          & El usuario debe poder visualizar el cuadro de mando a través de una página web \\
		\textbf{Precondición}         & Poseer la URL de la web \\
		\textbf{Acciones}             &
		\begin{enumerate}
			\def\labelenumi{\arabic{enumi}.}
			\tightlist
			\item Abrir el navegador web
			\item Introducir la URL del cuadro de mando
		\end{enumerate}\\
		\textbf{Postcondición}        & Ver el cuadro de mando \\
		\textbf{Excepciones}          & Caída de Azure \\ & Caída de Internet \\ 
		\textbf{Importancia}          & Alta \\
		\bottomrule
	\end{tabularx}
	\caption{CU-19 Publicar los resultados}
\end{table}