\capitulo{6}{Trabajos relacionados}

En este capítulo se presentan los trabajos y artículos relacionados con el contenido del proyecto que han sido de mayor relevancia para su desarrollo.

\begin{itemize}
    \item \textbf{Trabajo de Fin de Grado. Inteligencia Artificial aplicada a Destinos Turísticos Inteligentes:} Este proyecto nace de un trabajo de fin de grado realizado por Isabel Marcilla Lombraña para la Universidad de Burgos en 2024. El trabajo se centra en el uso de la inteligencia artificial para mejorar la experiencia del usuario en destinos turísticos inteligentes, y proporciona una base sólida para el desarrollo de este proyecto. Precisamente, el principal objetivo era tratar de llevar el desarrollo del mismo más lejos automatizando la recogida de reseñas que se hacía manualmente. También fue de gran ayuda para determinar algunas categorías a diferenciar en este proyecto por el modelo de clasificación\cite{tfg-2024}
    \item \textbf{FREE Google Maps Reviews Scraper API - How to Scrape Google Reviews:} Este vídeo tutorial de YouTube explica como utilizar el actor Google Maps Review Scraper de Apify. Al inicio del proyecto cuando estuve buscando formas de extraer las reseñas fue de gran ayuda debido a su bajo coste respecto a otros y a la posibilidad de ejecutarlo directamente desde Python. \cite{youtube:apify}
    \item \textbf{Documentación de la API de Apify:} La documentación oficial de la API de Apify fue esencial para entender como desarollar el código que interactúa con la API y extrae las reseñas de Google Maps. Proporciona ejemplos claros y una guía completa sobre cómo utilizar sus servicios. \cite{apify:docs} 
    \item \textbf{Documentación de Google Maps Places API:} Al principio del proyecto estuve probando a extraer las reseñas directamente desde la API de Google Maps. Para ello, me basé en la documentación oficial de Google Maps Places API y estuve probando varias formas y funciones de lograr mi objetivo. A pesar de que finalmente no fue la opción elegida para extraer las reseñas debido a su elevado coste, sirvió para un paso intermedio en el que se extraen los placeIds para utilizarlos posteriormente en la API de Apify. \cite{googleplaces:docs}
    \item \textbf{Documentación de Tensorflow Keras:} Durante la creación del modelo de clasificación de reseñas, la documentación de Tensorflow Keras fue de gran ayuda para comprender la sintaxis a utilizar. Además, proporciona ejemplos y guías sobre cómo construir y entrenar modelos de aprendizaje automático, lo que facilitó el proceso de implementación del modelo de clasificación. \cite{tensorflow:keras}
\end{itemize}
