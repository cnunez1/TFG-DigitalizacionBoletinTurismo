\capitulo{7}{Conclusiones y Líneas de trabajo futuras}

Tras finalizar el desarrollo del proyecto, es importante reflexionar sobre los resultados obtenidos y las lecciones aprendidas. Estas reflexiones no solo ayudan a evaluar el éxito del proyecto, sino que también proporcionan una base para futuras mejoras y desarrollos. 
En este capítulo se presentan las conclusiones generales del proyecto y se realiza un análsis crítico sobre el mismo para proponer posibles líneas de trabajo futuras.

\section{Conclusiones}

Puedo decir que el proyecto ha tenido un éxito parcial en la consecución de sus objetivos. Esto se debe a que se han logrado implementar todas las funcionalidades requeridas pero no de la forma más óptima. 
Esto se debe principalmente al elevado costo que supone el desarrollo de un proyecto de estas características. Lograr la automatización de las reseñas fue un gran desafío. Extraer tal cantidad de reseñas desde una API requiere un elevado costo económico y temporal lo que ha limitado la magnitud del proyecto.
A pesar de ello, se ha conseguido un producto funcional que cumple con los requisitos establecidos al inicio del proyecto.

A pesar de las limitaciones, puedo decir que he aprendido mucho durante el desarrollo del proyecto. He aprendido a trabajar con APIs y he adquirido experiencia en el manejo de datos. Además, he tenido la oportunidad de aplicar mis conocimientos teóricos en un proyecto práctico de mayor escala a lo que estoy acostumbrado, lo que ha sido muy enriquecedor.
He podido aprender a trabajar con bibliotecas como Keras, que es una de las más utilizadas en el campo del aprendizaje automático o a documentar con \LaTeX, que es una herramienta muy útil para la creación de documentos técnicos y científicos.
También cabe destacar que he aprendido a organizar el desarrollo de un proyecto software y a gestionar las tareas de desarrollo durante el proyecto.

Creo que en general ha sido una tarea ardua y compleja pero enriquecedora. Algo que no se muestra en el proyecto es el tiempo invertido en la investigación y pruebas de diferentes opciones descartadas hasta llegar a la solución final. Puedo decir con casi total seguridad que ese tiempo ha sido mayor que el tiempo invertido en el desarrollo del proyecto final en sí.

\section{Líneas de trabajo futuras}

A pesar de que el proyecto ha sido un éxito parcial, hay muchas áreas que se pueden mejorar y desarrollar en el futuro. Algunas de las posibles líneas de trabajo futuras incluyen:

\begin{itemize}
    \item \textbf{Mejorar la extracción de datos:} Encontrar una forma de extraer las reseñas de forma más eficiente y completa. Debido al elevado coste económico que conlleva esto ha sido muy difícil lograr extraer las reseñas. Esto podría lograrse mediante alguna API futura. Es importante considerar la legalidad de estas técnicas y el fin del proyecto.
    \item \textbf{Ampliación del conjunto de datos:} Ampliar la base de datos para incluir más subcategorías y obtener un conjunto de datos más diverso. Esto podría mejorar la generalización del modelo y su capacidad para hacer predicciones más precisas. Dependiendo de la ampliación de los datos, incluso podría cambiar el modelo de la base de datos.
    \item \textbf{Mejora del modelo de aprendizaje automático:} Se pueden explorar diferentes arquitecturas de redes neuronales y técnicas de optimización para mejorar la precisión del modelo.
    \item \textbf{Mejora del sistema de recomendación:} Desarrollar una mejora del sistema de recomendación más robusto que sugiera recursos a los usuarios en función de sus preferencias y del análisis de las reseñas. Esto podría mejorar la experiencia del usuario y aumentar la satisfacción de los potenciales viajeros o clientes interesados.
\end{itemize}